\documentclass[12pt]{article}

\usepackage{fullpage}
\usepackage{amsfonts}
\usepackage{epsfig}

\begin{document}

\title{Optimization Subject to Hidden Constraints via Statistical Emulation}
\author{
  Herbert K.H.~Lee\\
  Applied Math \& Statistics\\
  University of California, Santa Cruz\\
  {\tt herbie@ams.ucsc.edu} \and
  Robert B.~Gramacy\\
  Booth School of Business\\
  University of Chicago\footnote{Much of this work was done while
    R. B. G. was at the Statistical Laboratory, University of Cambridge}\\
  {\tt rbgramacy@chicagobooth.edu} \and
  Crystal Linkletter\\
  Center for Statistical Sciences\\
  Brown University\\
  {\tt cdlinkle@stat.brown.edu} \and
  Genetha A.~Gray\\
  Predictive Simulation R\&D\\
  Sandia National Laboratories\\
  {\tt gagray@sandia.gov}}

\maketitle

\begin{abstract}
\noindent
We present new methodology for constrained optimization based on
building a combination of models, one for the objective function and
one for the constraint region.  We use a treed Gaussian process as a
statistical emulator for the complex objective function, and a random
forest to model the probability of meeting the constraints.  By
combining these models, we can guide the optimization search to
promising areas in terms of both the objective function and the
constraint.  This approach avoids the problem of becoming stuck in a
local mode, as well as being able to deal with unconnected viable
regions.  We demonstrate our methodology on a simulated problem and an
example from hydrology.

\bigskip
\noindent {\bf Key words:} constrained optimization, surrogate
model, Gaussian process, sequential design, expected improvement

\bigskip
\noindent {\bf Mathematics Subject Classification:}  62M30 62C10 49M30 90C56
\end{abstract}

\renewcommand{\baselinestretch}{1.9}\small\normalsize

\section{Introduction}

Optimization of complex functions, where many separated local optima
abound, is a challenging problem with a long history.  Further
complication arises when the function can only be successfully
evaluated over a subset of the input space, and this region is not
known in advance. In this case, it might also be of interest to learn
about the feasibility region itself.  Our motivating example comes
from hydrology, where we are using a computer simulation for designing
the placement of a set of wells. We want to minimize the cost of the
setup.  However, for most runs of the simulator, the response is that
there has been a constraint violation, and no further information
about cost is provided.  Thus we need to simultaneously learn about
the constraint region and solve the minimization.  A further
complication is that no derivative information may be available, so we
are restricted to derivative-free optimization methods.

We note that constraints come in several kinds.  When the constraints
are known restrictions on the valid set of inputs, it is reasonably
straightforward to focus the optimization on only the valid set.
However when the constraints depend on an evaluation of the function,
either on values of the outputs or a failure during the evaluation, the problem
becomes considerably more difficult (for example \cite{fink:kell:2009}
and the references therein).  We further distinguish between
physical or hidden constraints, where it is not possible to get an
output value, and policy or unknown constraints, where an output value is
obtained and then deemed not valid.  In this paper we focus on the
former case, where the computer simulation fails to return a value for
some runs.  In that case, no information about the objective function
can be learned from a run outside the valid set, and one wants to
learn the boundaries of valid regions to avoid wasting runs on input
settings outside that set.

Our approach is to use statistical emulation, where we build a
statistical model as an approximation to the complex simulator, and
use this model to guide the optimization.  This model has two parts:
one to predict the response surface when the simulator gives a valid
response, and one to predict whether or not the simulator will return
a valid response at all.  We use treed Gaussian processes
\cite{gra:lee:2008} for response surface prediction, and random
forests \cite{brei:2001} for constraint violation prediction.  The
former is becoming widely recognized as a flexible surrogate model, or
{\em emulator}, for the design and analysis of computer experiments.
The latter is a powerful classification algorithm from the machine
learning literature.

In Section~\ref{sec:models} we review the statistical emulation
techniques.  In Section~\ref{sec:opt} we apply the statistical
predictions to optimization using expected improvement.  In
Section~\ref{sec:ex} we then
provide an illustrative example before finally applying our methods to
the motivating example in hydrology in Section~\ref{sec:hydro}.


\section{Statistical Emulation} \label{sec:models}

The use of stochastic or statistical models to approximate a complex
function, such as the output of a computer simulator, is now well
established in the literature
\cite{sack:welc:mitc:wynn:1989,kennedy:ohagan:2000,sant:will:notz:2003,fang:li:sudj:2006}.
In particular, the standard model used for emulation is a Gaussian
process (GP).  The functional response is treated as a random variable
$Z({\bf x})$ that depends on the input vector ${\bf x}$ such that the
response varies smoothly.  The degree of smoothness is determined by
the covariance structure of the GP. GPs have the property that any
finite set of locations has a joint multivariate normal (Gaussian)
distribution, thus the process is completely determined by its mean
and covariance function. If we have observations at a set of locations
${\bf x}_1, \ldots, {\bf x}_n \in \mathcal{X}$, then
\begin{equation}
\left(Z({\bf x}_1), \ldots, Z({\bf x}_n)\right) \sim \mathrm{MVN}\!\left(
  \mbox{\boldmath{$\mu$}}, \mbox{\boldmath{$\Sigma$}} \right) ,
\end{equation}
where $\mbox{\boldmath{$\mu$}} = \left(\mu({\bf x}_1), \ldots,
  \mu({\bf x}_n)\right)$ is a mean function and {\boldmath $\Sigma$}
is the variance-covariance matrix.  The mean function is typically
taken as constant, linear, or a low-order polynomial.  The covariance
is typically given a simple parameterization such that correlation
decreases with distance in the input space.  More details on GPs are
available in references such as \cite{cres:1993} and
\cite{stei:1999}.  We take a fully Bayesian approach, allowing for
estimation of uncertainty, which is critical when trying to determine
the probability that an unsampled location will be an improvement over
the current known optimum.

In some cases standard GP models are adequate.  But in others, their
limitations, such as strong assumptions of stationarity and poor
computational scaling, can be problematic.  To reduce these problems,
we instead use treed Gaussian process (TGP) models
\cite{gra:lee:2008}.  The basic idea is to use a recursive tree
structure to create axis-aligned partitions that segment the space
into regions where the assumption of stationarity is valid, and then
fit independent GP models within each partition.  The approach is an
extension of the partition models of \cite{chip:geor:mccu:2002}.
Operating under the Bayesian paradigm, the partitions can be fit
simultaneously with the parameters of the embedded GP models using
reversible jump Markov chain Monte Carlo \cite{gree:1995}.  Fitted
values and predictions are obtained by averaging over the posterior
distribution for the tree structure, so that the model can produce
smooth fitted functions.  There is software available in the {\tt tgp}
library \cite{Gramacy:2007} for the open source statistical package
{\sf R} \cite{R.base}.
%(see {\tt http://www.cran.r-project.org/src/contrib/Descriptions/tgp.html}).

In addition to emulating the simulator, we also need to predict
whether the simulator will return a valid response or whether there
will be a constraint violation.  For this task we turn to a machine
learning tool, random forests. At each training input ${\bf x}$, we
record a dichotomous variable $Y({\bf x})=0$ if the simulator fails to
return a valid response, and $Y({\bf x})=1$ if a valid response $Z({\bf
  x})$ is returned (which is used in the construction of the emulator
as described above).  For example, when $n$ runs are completed, the
binary data $Y({\bf x}_1),\ldots,Y({\bf x}_n)$ are used to train a
collection of $B$ CART trees, each randomly instantiated
\cite{brei:1984} using $m$ randomly selected splitting variables.  We
use the average vote over the $B$ trees to obtain the predicted
probability that an untried input location $\bf{x}$ will return a
valid response: $h({\bf x}) =
\frac{1}{B}\sum_{b=1}^B\hat{Y}_b(\mathbf{x})$, where
$\hat{Y}_b(\mathbf{x})$ is the classification assigned in the $b^{th}$
randomly started tree.  It has been shown that this process converges
asymptotically as the number of trees grows large, but that in
practice even moderately sized choices for $m$ and $B$ provide a good
approximation.  More details on random forests are available in
\cite{brei:2001} and comparison with other classification methods is
discussed in \cite{hastie:tibshirani:friedman:2001}.  We use the {\sf R}
implementation from the {\tt randomForest} library
\cite{liaw:wien:2002}.



\section{Optimization} \label{sec:opt}

Our approach toward optimization is that of expected improvement (EI)
\cite{jones:schonlau:welch:1998,tadd:lee:gray:grif:2009}.  We
use the statistical emulator, or surrogate model, to guide the search
for a new optimum, sequentially investigating locations with the
largest probability of being an improvement.  To be concrete, we focus
on the case of minimization (but it can clearly be applied to
maximization as well).  Herein we assume that the functional response
is deterministic, but the method can be generalized for stochastic
response functions as well.  After $N$ runs, denote the current
minimum value observed as $f_{\min} = \min\{z_1,\ldots,z_N\}$, where
$z_i = Z(\mathbf{x}_i)$.  Define the improvement statistic at a
proposed input location ${\bf x}$ by
\begin{equation} 
I({\bf x}) = \max\{f_{\min} - Z({\bf x}), 0\}.
\end{equation}
Since the ${\bf x}$ of interest is the vector of previously unobserved locations,
$Z({\bf x})$ is unknown and we represent its distribution using
the posterior predictive distribution from the treed Gaussian process
emulator.  A simple algorithm for optimization sequentially
selects the points ${\bf x'}$ that maximize the expected improvement (EI)
\begin{equation}
{\bf x'} = \mbox{arg} \max_{{\bf x}\in \mathcal{X}} \mathbb{E}\{I({\bf x})\}.
\label{eq:ei}
\end{equation}
The EI provides a combined measure of how promising a candidate point
is, balancing points where the emulator predicts a minimal response
with those points where the uncertainty in the emulator is large. For
the latter locations, there is a probability of a minimal response
even though the mean prediction may not be as small due to the
emulator uncertainty.

Our approach for evaluating the EI is with our TGP emulator.  As with
most problems in Bayesian statistics, we use Markov chain Monte Carlo
(MCMC) \cite{gelm:carl:ster:rubi:1995} to fit the emulator, obtaining
a Monte Carlo sample of treed Gaussian processes from the posterior
distribution.  For each of these posterior samples, the expectation
calculation (\ref{eq:ei}) is available in closed form as a function of
the mean and variance of the (Gaussian) posterior predictive
distribution of $Z(\bf x)$.  We then obtain a Monte Carlo estimate of
the posterior EI by taking an average of the corresponding EI
calculations that arise from each sample.  Our implementation takes
advantage of the functionality of the {\tt tgp} {\sf R} library, which
provides an argument for evaluating the expected improvement,
essentially automating the EI calculation.

In the case where there are no constraints, one can just iteratively
evaluate the next point that maximizes the EI (as estimated from the
emulator), and then update the statistical emulator and the
estimated improvement function.  There are several 
algorithms that may be used to aid in the search of the $\bf x$ input
that maximizes the EI.  In the case of a simple (non-treed) GP
surrogate model and maximum likelihood inference, there is a branch
and bound algorithm that may be used
\cite{jones:schonlau:welch:1998}.  This leads to the so-called
expected global optimization (EGO) algorithm.  The more sophisticated
and fully Bayesian TGP model requires a more sophisticated search
method.  One useful construct is a Latin hypercube design (LHD)
\cite{mcka:cono:beck:1979} which is a space-filling design of
$n$ points that divides the $p$-dimensional input space into an $n
\times n \times \ldots \times n = n^p$ grid and then arranges to have
exactly one sample in each row in each dimension.
\cite{tadd:lee:gray:grif:2009} used the TGP EI calculations
on random LHDs of candidates as an ``oracle'' sub-routine
within a direct/pattern-search optimizer called APPS
\cite{gray:kold:2006,kold:2005} and illustrate how these tools may be
used to obtain the optimal design for a circuit device.
Gramacy and Taddy \cite{gramacy:taddy:2010} (Section 3) propose a simpler, {\sf
  R}-centric variant via the opposite embedding---where the LHD
candidates are augmented by an oracle point obtained by searching the
maximum {\em a posteriori} TGP surface for minima via the {\tt optim}
function---that has been shown to perform well in many examples.

In any case, the presence of constraints complicates the notion of
expected improvement.  It does not make any sense to knowingly waste
time evaluating the function in a region expected to return a
constraint violation.  Thus we need to trade off between the EI and
the probability that a valid response will be returned.  To do this,
we simply multiply the EI by the probability that we will get a valid
response. At each tried location, we can record whether or not a valid
response was returned by the simulator. This information can be used
to inform a random forest classifier. Let $h({\bf x})$ be the
predicted probability of a valid response at input ${\bf x}$ estimated
by the random forest classifier.  We extend the algorithm in Equation
(\ref{eq:ei}) by now choosing the point ${\bf x'}$ that maximizes the
expected constrained improvement:
\begin{equation}
{\bf x'} = \mbox{arg} \max_{{\bf x}\in \mathcal{X}} \mathbb{E}\{I({\bf
  x})\}h({\bf x}).
\label{eq:eci}
\end{equation}
With this modification, we can focus our efforts where they are most
likely to be fruitful.  How much (\ref{eq:eci}) differs from
(\ref{eq:ei}) will depend on the shape of the constraint boundary.
Note that this approach takes a global view of 
the optimization problem, as both the function emulator and the
constraint probability map cover the full input space.  As with
vanilla EI, there are several ways one can proceed to search for the
${\bf x}$ which maximizes the EI.  All of the techniques mentioned
above would apply in the expected constrained improvement context.
Our approach is based on the {\sf R} implementation described by
\cite{gramacy:taddy:2010}.

In summary, an outline of our implementation of this algorithm is:
\begin{enumerate}
\item Generate a small Latin Hypercube Design of points and evaluate
  the function at those points, determining the validity of responses
\item Repeat the following until convergence or computing budget exhausted:
 \begin{enumerate}
 \item Fit the TGP emulator and random forest classifier
 \item Compute the EI and the expected constrained improvement
  (\ref{eq:eci}) over a set of candidate locations (such as from a LHD)
 \item Find the candidate which maximizes the expected constrained improvement
 \item Evaluate this point and determine if its response is valid, and
  add it to the set of evaluated points
 \item If the response is valid, move to the next iteration.
 Otherwise, continue to find the candidate with the next highest expected
 constrained improvement and evaluate it until a valid point is found,
 adding all points to the training set for the random forest classifier
 \end{enumerate}
\end{enumerate}
For convergence we can monitor the magnitude of the maximum expected
constrained improvement and terminate when it is smaller than a
threshold value.  In practice, we are often faced with a limited
computing budget for evaluating the complex objective function, and
may need to simply terminate the search when our computing budget is
exhausted. 

\section{Illustrative Example} \label{sec:ex}

We provide a relatively simple example here to illustrate our
methods.  Suppose we want to minimize the function
\begin{eqnarray}
Z(x_1,x_2) & = & -w(x_1)w(x_2) \\
w(x) & = & \exp\left(-(x-1)2\right) + \exp\left(-0.8(x+1)2\right) 
- 0.05\sin\left(8(x+0.1)\right).
\end{eqnarray}
Figure~\ref{fig:ex1.uncon} shows the negative of the surface (i.e., as
a maximization problem) for visibility (it is much easier to see the
peaks than the valleys).  There are four separated modal regions, and
local modes within each of the regions.  The true global minimum is
shown at $x_1=-1.0408, x_2=-1.0408$.
\begin{figure}[ht!]
\begin{center}
\includegraphics[scale=0.55, trim=0 0 0 0]{wobble_plot}
\end{center}
\caption{The function to be optimized, shown without constraints and
  shown as a maximization problem for visibility.  The optimum is
  shown with the dot on the right hand plot.}
\label{fig:ex1.uncon}
\end{figure}
Now suppose that when we try to evaluate this function, it will only
return valid values inside of an ellipse, but that we don't know
anything about this valid region in advance and need to learn about it
from the data.

To begin our search, we generate an initial design of 20 random points
from a LHD and evaluate the function at those points.  We use this
initial dataset to fit the treed Gaussian process emulator and the
random forest classifier.  We then compute the EI at a fresh set of
100 LHD locations, multiply it by the predicted probability of
obtaining a valid response at those candidate locations, and choose
the maximum of that result as the next point to evaluate.  If we
obtain a valid response, that completes the iteration. Otherwise, we
sample at the location with the next highest expected constrained
improvement until we obtain a valid response (note that until a valid
response is obtained, there is no need to update the emulator).  We
only retrain the random forests classifier and TGP emulator 
once per iteration, i.e., only after a valid response is obtained.  We
do fifty iterations, which provides us with the answer.

Figures~\ref{fig:ex1.20} and \ref{fig:ex1.30} show the results from
this implementation.  Figure~\ref{fig:ex1.20} shows an intermediate
stage after twenty iterations.  The upper left plot shows the truth
along with the bounding ellipse (only points inside the ellipse return
a valid value) which we are treating as unknown and
arbitrarily-shaped.  The lower left plot shows the current emulated
surface, which is a reasonable approximation of the truth within the
valid region, although it will continue to improve as more locations
are sampled (and no data is available outside the valid region, so we
cannot expect a perfect match outside that region).  The solid dots
show the locations that have been evaluated, and the open circles show
the LHD over which we conduct our search for the next point with
highest expected constrained improvement.  The top middle plot shows
the EI surface.  The bottom middle plot shows the predicted
probability of returning a valid value obtained from fitting the
random forest classifier.  The open circles are the locations which
have been evaluated successfully, and the filled circles are the
locations which have been evaluated but produced a constraint
violation.  The upper right plot shows the product of the EI and the
probability of being valid, and it is this surface that is maximized.
The location shown at the crosshairs is the chosen point.  The bottom
right plot shows the progress in minimization.  As the process
proceeds, the emulator and the classifier better learn about their
respective problems and seek out lower minima until they cannot find
anything better.  Figure~\ref{fig:ex1.30} shows the result after
thirty iterations (67 function calls).  The minimum was found after 39
function evaluations in this run.

\begin{figure}[htb]
\begin{center}
\includegraphics[scale=0.55, trim=0 0 0 0, angle=270]{nanplot20.ps}
\end{center}
\caption{After 20 iterations: (a) the function to be optimized with valid
  region, (b) the fitted emulator, (c) the EI surface, (d) the
  fitted probability of being valid, (e) the expected constrained
  improvement, and (f) the progress in minimization.  White shows high
  values and green (dark) shows low values.}
\label{fig:ex1.20}
\end{figure}

\begin{figure}[htb]
\begin{center}
\includegraphics[scale=0.55, trim=0 0 0 0, angle=270]{nanplot30.ps}
\end{center}
\caption{After 30 iterations: (a) the function to be optimized with valid
  region, (b) the fitted emulator, (c) the EI surface, (d) the
  fitted probability of being valid, (e) the expected constrained
  improvement, and (f) the progress in minimization.  White shows high
  values and green (dark) shows low values.}
\label{fig:ex1.30}
\end{figure}

For comparison, we pit our approach against a genetic algorithm (GA).
The GA was picked as it is another global approach which does not
depend on having a good starting location, whereas most local methods
require a good staring point with which to initialize the algorithm.
We ran our procedure 100 times for 100 iterations each, and we also
ran a GA implementation in R \cite{genalg} 100 times with a
population size of 10 for 50 iterations, for a total of 500 function
evaluations.  On average, our approach required 137.2 function
evaluations, and found an average minimum value of -1.0904, whereas
the GA was budgeted 500 function evaluations for each run but only
found an average minimum value of -1.0864.  Our method was within .005
of the true minimum 84\% of the time, while the GA was within .005
only 58\% of the time, despite using many more function evaluations.


\section{Hydrology Example} \label{sec:hydro}

A key motivating example for this work is the hydraulic capture
problem from the community problems \cite{maye:kell:mill:2002}.  The
goal of the problem is to find a configuration of up to four wells to
control the direction of groundwater movement to contain a contaminant
plume, and to do so at minimal cost.  The constraints in this problem
are in the form of specifications on the hydraulic head differences at
designated locations, and the objective function is the monetary cost
of installing and operating the wells.  More details on this problem
are available in \cite{maye:kell:mill:2002} and
\cite{fowl:kell:kees:mill:2004}.  A number of solutions have been
proposed for this problem, and some comparisons and discussion of the
difficulties are available in references such as
\cite{fowl:2008,gray:fowl:grif:2009,HemkerMINP}.  This problem is
particularly challenging for two reasons.  First, the objective
function is quite irregular and difficult to predict.  Second, the
dimension of the space is not fixed, varying from twelve dimensions if
four wells are used, down to three dimensions if only one well is used
(the three parameters corresponding to each well are detailed below).

The hydraulic capture problem is described in
\cite{maye:kell:mill:2002} and the implementation details can be found
in \cite{fowl:2008}.  Intuitively, the problem is illustrated in
Figure~\ref{fig:HCidea}.  The green oval shows the relative location
of a contaminant plume in a rectangular aquifer. The circles around
the exterior of the plume represent the gradient constraints on the
hydraulic head values, aligned so that flow will be forced towards the
interior of the plume to prevent migration.  Groundwater flow in the
natural system in the absence of wells is towards the northeast,
indicated by the arrow in the lower left corner. In this illustration,
we also use stars to show a reasonable configuration of four
wells---two extracting water inside the plume and two injecting water
outside the plume.  The inputs of the objective function are the well
locations (in x-y coordinates) and the flow rates of each of the
wells, so that there are three parameters corresponding to each well.
The function output is the total cost of installing and operating the
wells.  Note that the calculation of this cost requires the results
from the groundwater simulator.

\begin{figure}[htb]
\begin{center}
\includegraphics[scale=0.6, trim=0 0 0 0]{HCidea.ps}
%\psfig{figure=HCidea.ps,width=2.0in,angle=0}
\end{center}
\caption{Pictorial representation of the constrained optimization problem}
\label{fig:HCidea}
\end{figure}

Our first approach is to start with four wells and allow the search to
work its way down to a single well iteratively.  When the flow rate of
a well is less than $1 \times 10^{-4}$, the well is turned off and the
flow rate is set to zero.  The search is initialized by seeding the
space with a LHD to obtain some points to fit the emulator.  We then
propose a set of candidate points from a fresh LHD, but we augment the
candidate set over which we evaluate the expected improvement in two
ways.  First, following \cite{tadd:lee:gray:grif:2009}, we add fifty
points in a ball of small radius around the best point, which helps in
exploring higher-dimensional spaces.  Second, we add a candidate point
equal to the best point with the smallest well rate set to flow zero.
If this point is determined to be the best, it allows us to remove a
well from the design and move to a lower-dimensional problem.  We then
fit the TGP emulator and the random forest classifier, and rank the
candidate points based on the product of their expected improvement
and probability of being valid, as per Equation~(\ref{eq:eci}).  We
evaluate the points in rank order until one returns a valid output,
then we move to the next iteration, updating the statistical models
and repeating.  Convergence can be assessed by tracking the maximum
expected constrained improvement.  This algorithm is able to easily
reduce the number of wells from four to two, but it has more
difficulty moving to a single well solution.  The best solution it
finds is a cost of \$36,389.83 using two wells. We note that this
solution is better than the solutions found by six of the nine methods
used in a comparison study of techniques on this problem by
\cite{fowl:2008}.

However, this two well solution is not the best solution.  We take a
second approach to searching the space, using our expected improvement
approach for each of one-, two-, three-, and four-well spaces
separately.  In this case, we do not supplement the candidate search
with a point with fewer wells, nor do we remove wells with extremely
small flow rates.  With this fixed-dimension approach, we find a best
one-well solution of \$22,952.77, which is better than all of the
solutions found in \cite{fowl:2008}, with the best solution reported
there being \$23,421.  We note that the setup of the experiment here
is different than in the alternatives; our method requires more than
one initial valid point to initialize the statistical models. The
comparison methods might perform differently under the same initial
conditions we used here.  For a small number of possible wells, it is
more efficient for us to search each space separately, as the
statistical models can train more effectively to each state space, but
this approach would not scale as well to larger dimensional spaces.


\section{Conclusions} \label{sec:concl}

In conclusion, we have proposed a new approach for constrained
optimization based on statistical models.  By simultaneously learning
about both the objective function and the boundary of the constraints,
we are able to better focus our efforts on the valid regions.  We note
that because our methods are based on statistical models, they also
apply to the cases of stochastic simulators or probabilistic constraints.

Some areas for further work include the consideration of other
classification models.  For example, when the constraint boundary is
relatively smooth, a Gaussian process classifier \cite{brod:gram:2011} 
may learn it more quickly.  It would be worthwhile to compare the
relative efficiency of different classifiers under different types of
feasible regions (e.g., connected versus disjoint feasible regions).
Another possibility is to connect the surrogate model with the model
for estimating the constraint boundary.  A different approach is to
move beyond expected improvement at a point and consider a more global
improvement function, such as one that integrates improvement over the
whole space.  Convergence diagnostics could also be better explored,
such as those along the lines of \cite{gramacy:taddy:2010} (Section 3).

\subsection*{Acknowledgments}

This research was initiated at a workshop at the American Institute of
Mathematics on Derivative-Free Hybrid Optimization Methods for Solving
Simulation-Based Problems in Hydrology, and was also partially
supported by NSF grant DMS-0906720 to HKHL and EPSRC grant
EP/D065704/1 to RBG.  We thank the reviewer and associate editor for
their helpful comments.

\bibliographystyle{plain}
\bibliography{intei}

\end{document}
