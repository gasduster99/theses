\documentclass[12pt]{article}

\usepackage{graphicx}
\usepackage[margin=1in,footskip=0.2in]{geometry}%big footskip brings number down, small footskip brings number up
\usepackage{amsmath}
\usepackage{amssymb}
\usepackage[T1]{fontenc}
\usepackage{listings}
\usepackage{bm}
\usepackage{hyperref}
\usepackage{setspace}
\usepackage[usenames]{color}
\usepackage[utf8]{inputenc}


\hypersetup{colorlinks   = true, %Colours links instead of ugly boxes
            urlcolor     = black, %Colour for external hyperlinks
	    citecolor    = blue,
	    linkcolor    = black
}

\renewcommand*\contentsname{Table of Contents}


%\usepackage{multimedia} 
%\usepackage[noplaybutton]{media9}


%draws a line across the page
\newcommand{\E}[1]{
        \mathbb{E}\left[#1\right]
}

\def \EI {
	\mathbb{E}\left[~\mathcal{I}~\right]
}

\def \EIx {
	\mathbb{E}\left[~\text{I}(\bm{x})~\right]
}



\begin{document}
\pagenumbering{gobble}
$~$
%\newline
%\newline
%\newline
\begin{center}
{\singlespace
UNIVERSITY OF CALIFORNIA\\
SANTA CRUZ\\}
$~$\\
{\bf IDENTIFYING CONVERGENCE IN GAUSSIAN PROCESS SURROGATE \\MODEL OPTIMIZATION, VIA STATISTICAL PROCESS CONTROL\\}
$~$\\
{\singlespace
A document submitted in partial satisfaction of the\\
requirements for the degree of\\}
$~$\\
MASTER OF SCIENCE\\
$~$\\
in\\
$~$\\
STATISTICS AND APPLIED MATHMATICS\\
$~$\\
by\\
$~$\\
{\bf Nicholas R. Grunloh\\}
$~$\\
July 2014\\
\end{center}

$~$\\
\begin{table}[h!]
  \raggedleft
  \vskip.5in
  \begin{tabular}{l l}
    Approved by:
    \\\\\\\\
    \hline
    \\
    ~~~~~~~~~~~~~~~Professor Herbert Lee~~~~
    \\\\\\\\
    \hline
    \\
    ~~~~Associate Professor John Musacchio~~~~
  \end{tabular}
\end{table}

\clearpage
\tableofcontents
\pagenumbering{roman}
\clearpage
\pagenumbering{arabic}

\begin{abstract}
\begin{verbatim}
__________
< Abstract Cow >
 -------------
        \   ^__^
         \  (**)\_______
            (__)\       )\/\
             U  ||----w |
                ||     ||
\end{verbatim}
\end{abstract}



% \includemovie[width=10cm]{rast20}
%\movie[height=0.6\textwidth,width=0.6\textwidth]{}{clt.gif}%Circle-m-increase3.mp4}

% {\color{red} Figure out which figures and where.} 
% {\color{red} Moving pictures?}

\section{Introduction}
	\begin{itemize}
	\item Identify convergence problem
	\item Define properties of convergence 
	\item A taste of my stuff
	\item Explicate road-map
	\end{itemize}
	%
	\subsection{General}
		%enormous best choice?
		Derivative-free optimization is an enormously practicle and commensurately difficult task.
		%gradient descent is very eager, perhaps too eager to find optima
		Derivative information, when availiable, has the capacity to very efficently lead the user to an optimal solution, as well as signal to the user that such an optimum has been found.
		%
		However, derivative information is not always availiable.
		%we are left quite litterally no knowing whichc way is up :)
		In these cases optimization is just as desirable of a task, but in the absence of derivatives we are left to find more creative ways of figuring out which way is up.
		%
		Examples of this creativity can be seem in the diversity of some of the more popular methods of derivative-free optimazation.
		 
		%evolutionary/genetic
		Examples of various effective derivative-free optimazation stategies include: evolutionary algorithms (EA), pattern search (PS) alorithms, restricted-step/trust region methods, as well as simulated annealing (SA). 
		%
		Fundamentally these methods share 3 basic components.
		%new candidate points for evaluation of potentially optimal points.
		Firstly, there is some proceedure for exploring the objective function space. 
		%
		Secondly, these methods must use information from these exploratory function evaluations to update the exploratory proceedure, and thus, explor more effectively.
		%
		Thirdly, they are tasked with identifying when they have found an optimal point.
		%
		By tweaking any one of these components the optimization behaviour, with respect to scope, convergence, and accuracy, mat differ dramatically. 
		
	\begin{itemize}
	\item Optimization: Background and our Philosophy
		\begin{itemize}
		%\item gradient free \cite{noGradBook}
		\item GA {\color{red} citation}, Simulated Annealing {\color{red} cite}, Pattern Search \cite{noGradBook}, Trust Regions \cite{noGradBook} 
		\item benefits of a model based approach (ie. uncertain measures)(i.e. General/No GP)
		\item uncertainty measures as convergence criteria
		\end{itemize}
	%	
	\item Statistical process control for monitoring convergence
		\begin{itemize}
		\item General \cite{shewhartBook} (ie. no EWMA)
		\item tease EWMA
		\end{itemize}
	%
	\end{itemize}
	%
	\subsection{Gaussian Process Models}
	\begin{itemize}
	\item Gaussian Process Surrogate Model
	\item How does a Gaussian process work? \cite{gpJasa}
	\item Add Flexibility through treed Partitioning \cite{tgp}
	\end{itemize}
	%
	\subsection{Convergence Criteria}
	\begin{itemize}
	\item Measures?
	\item Choose $\EIx$, why? \cite{gBook}, \cite{tgp2}
	\item some characteristics [bounded at 0], decreasing.
	\item which $\E{~\text{I}^g(\bm{x})~}$, ie. which $g$? \cite{gBook}
	\item maximum $\EIx$ (i.e. the mean at the predictive location that achieves the maximum mean of the samples at that location)
	\end{itemize}
	%
	\subsection{Optimization}
	\begin{itemize}
	\item Optimization Procedure \cite{tgp2}
		\begin{itemize}
		\item code appendix, using tgp
		\end{itemize}
	%
	\item advantages of model based approach for convergence sake
	\item $\EIx$ Behavior for convergence
	%\item taste of SPC
	\end{itemize}
	%
	\subsection{Statistical Process Control}
		\subsubsection{Shewhart's $\bar{x}$ Chart}
		\begin{itemize}
		\item the notion of control (draw similarities to convergence). \cite{shewhartBook}
		\item how the typical charts work
		\item philosophy.
			\begin{itemize}
			\item establish control (herbies book)
			\item control $\rightarrow$ out-of-control 
			\end{itemize}
		\item stumbling blocks of convergence for me.
			\begin{itemize}
			\item out-of-control $\rightarrow$ control
			\item the notion of a sliding average (i.e. convergence)
			\item normal assumptions are very strong for an application that strongly desired robustness in varied applications% {\color{red} list them}
			\end{itemize}
		\end{itemize}
		%
		\subsubsection{Exponentially Weighted Moving Average Chart}
		\begin{itemize}
		\item EWMA philosophy (Robustness) \cite{boxBook}.
		\item How it works ({\color{red}derivation cite}).
		\item look at the statistics and bounds
		\item Tracking slight changes (general scale and behavior of $\EIx$) 
		\item weight recent data more heavily to handle the sliding average (also mention later about the window; maybe set-up here)%and why thats better for this problem.
		\end{itemize}
		%
\section{Identifying Convergence}
	\begin{itemize}
	\item tie this stuff together and motivate the coolness factor. 
	\item use optimization procedure outlined \cite{tgp2} also above and in appendix.
% 	\item look at the maximum $\EIx$. why? ({\color{red}maybe \cite{gBook}, \cite{tgp2}})
	\item recall that the maximum $\EIx$ each iteration is the mean at the predictive location that achieves the highest mean value.  
	\item SPC is based on normality assumptions of the underlying sampling distribution.
	\item a thesis statement for the research that I did: SPC, EWMA, empirical predictive MCMC control limits, Log-Normal $\rightarrow$ model based limits.
	\end{itemize}
	\subsection{The Control Window}
		\begin{itemize}
		\item convergence formulated in the context of statistical process control has unique challenges since almost by definition max $\EIx$ starts in an out-of-control state then moves into a state of control the optimization routine approaches a state of convergence%. we need to establish a method for  
		\item typical SPC goes through an initialization process, in which, initially out-of-control observation are investigated and systematically accounted for to establish an initial state of control.
		\item introduce the window to automate this process and thus set the current control state at a window of the most recently observed values.
		\item window of size $w$, tuning parameter, thus partitioning the observations into points in the control window(i.e. control training set) and points outside of the control window(i.e. control test set)
		\item convergence rules: out-of-control in control test set and in-control in control training set.
		\item choosing $w$ for difficulty of problem. 
		\end{itemize}

	\subsection{$\bar{x}$ Chart}
		\begin{itemize}
		\item basic shewhart chart
		%\item equally weighted observations leads to false positive in identifying convegence since initial $\EIx$ may be very large.
		\item issues with robustness since you often see linear trends toward the lower control limit as max$\EIx$ seems to converge in probability to 0 from the positive direction.  
		\end{itemize}
	
	\subsection{Model-Based Transformation} %Robustness Transformation}
		\begin{itemize}
		\item Transformation
		\item $\bar{x}$ Chart
		\item equally weighted observations leads to false positive in identifying convergence since initial $\EIx$ may be very large.
		\end{itemize}
	
	\subsection{EWMA Chart}%\subsection{Control Limits}%\subsection{Model Based Limits}
		\begin{itemize}
		\item
		\end{itemize}
%
{\color{red} Example pictures above?, or keep most of the figures that I made in the below sections?, start Rosenbrok example early?:
\begin{itemize}
\item max$\EIx$, General Behavior (3 stages, not-converged(inital exploration), converging(pre-convergence) and converged(converged))
\item 
\end{itemize}
}
\section{Test Functions}
	{\color{red} use each example as an excuse to look at different things?}
	\subsection{Rosenbrock}
		\begin{itemize}
		\item write the general function down, focus on the 2-D case, code appendix ({\color{red}cite?})
		\item get a good looking window
		\item plot what function looks like in this window, perspective and heat plot,
		\item gaussian process fit perspective and heat plot {\color{red}(movie?), thumbnail:first converged picture}%convert -delay 100 90*.jpeg rast20.mp4
 		\item Simple $\EIx$ Pictures
 			\begin{itemize}
 			\item $max\EIx$/best Z (tell three stages convergence story)
 			\item hist of $max\EIx$ samples
 			\item {\color{red}Q-Q plot?}
 			\item $\bar{x}$ Chart
 			\end{itemize}
 		\item Transformed Pictures
 			\begin{itemize}
 			\item $max\EIx$/best Z
 			\item hist of $max\EIx$ samples
 			\item {\color{red}Q-Q plot?}
 			\end{itemize}
		\item discussion of results
		\end{itemize}
	%
	\subsection{Rastringin}
		\begin{itemize}
		\item write down function {\color{red} cite)}
		\item several mode window
		\item plot what function looks like in this window, perspective and heat plot,
		\item gaussian process fit perspective and heat plot {\color{red}(movie?), thumbnail:first converged picture}%convert -delay 100 90*.jpeg rast20.mp4
		\item Transformed Picture $max\EIx$/best Z
		\item discussion of results
		\end{itemize}
	%
	\subsection{Easom}
		\begin{itemize}
		\item write down function {\color{red} cite}
		\item Get reasonably flat window 
		\item plot what function looks like in this window, perspective and heat plot,
		\item gaussian process fit perspective and heat plot {\color{red}(movie?), thumbnail:first converged picture}%convert -delay 100 90*.jpeg rast20.mp4
		\item Transformed Picture $max\EIx$/best Z
		\item discussion of results
		\end{itemize}
	
	\subsection{Real Data}
		\begin{itemize}
		\item explore this data {\color{red} cite}
		\item get good lookin picture of objective function
		\item gaussian process fit perspective and heat plot {\color{red}(movie?), thumbnail:first converged picture}%convert -delay 100 90*.jpeg rast20.mp4
		\item Transformed Picture $max\EIx$/best Z
		\item discussion of results
		\end{itemize}

\section{Discussion}
	\begin{itemize}
	\item argument for a convergence criteria based on above results
	\item Robustness of EWMA \cite{boxBook}
	\item further research partitioned model idea.
	\end{itemize}
	
\section{Code Appendix}
	\begin{itemize}
	\item tgp optimization \cite{tgp2}
	\item qcc EWMA SPC \cite{qccPack}
	\item example implimentation (cite data)
	\end{itemize}
	
	
% \bibliographystyle{plain}%jasa}%
% \bibliography{msCite}

%\bibitem{tgpPack} {\color{red}cite tgp package?}
%{\color{red} probably should be the 1931 book} Shewhart, W. A., \& Deming, W. E. (1939). {\em Statistical method from the viewpoint of quality control}. Washington: The Graduate School, The Dept. of Agriculture.
\begin{thebibliography}{2}
\bibitem{gpJasa} R. B. Gramacy, \& H. K. Lee. (2008). {\em Bayesian treed Gaussian process models with an application to computer modeling}. Journal of the American Statistical Association, 103, 1119-1130.
\bibitem{tgp} R. B. Gramacy, (2007). {\em tgp: An R Package for Bayesian Nonstationary, Semiparametric Nonlinear Regression and Design by Treed Gaussian Process Models}. Journal of Statistical Software, 19(9), 1-46. %URL http://www.jstatsoft.org/v19/i09/.
\bibitem{tgp2} R. B. Gramacy, \& M. Taddy (2010). {\em Categorical Inputs, Sensitivity Analysis, Optimization and Importance Tempering with tgp Version 2, an R Package for Treed Gaussian Process Models}. Journal of Statistical Software, 33(6), 1-48. %URL http://www.jstatsoft.org/v33/i06/.
\bibitem{shewhartBook} W. Shewhart (1931). {\em Economic Control of Quality of Manufactured Product}. New York: D. Van Nostrand Company, Inc. 
\bibitem{noGradBook} A. R. Conn, K. Scheinberg, \& L. Vicente (2009). {\em Introduction to derivative-free optimization}. Philadelphia: Society for Industrial and Applied Mathematics/Mathematical Programming Society.
\bibitem{qccPack} L. Scrucca (2004). {\em qcc: an R package for quality control charting and statistical process control}. R News 4/1, 11-17.
\bibitem{boxBook} G. E. Box, A. Luceno, \& M. Paniagua-Qui\~{n}ones, (1997). {\em Statistical control by monitoring and feedback adjustment}. New York: Wiley.
\bibitem{gBook} M. Schonlau, D. R. Jones, \& W. J. Welch (1998). {\em Global versus local search in constrained optimization of computer models}. In {\em New developments and applications in experimental design}, number 34 in IMS Lecture Notes. Monograph Series, 11-25.
\end{thebibliography}
%   Robert B. Gramacy, Matthew Taddy (2010). Categorical Inputs,
%   Sensitivity Analysis, Optimization and Importance Tempering with tgp
%   Version 2, an R Package for Treed Gaussian Process Models. Journal of
%   Statistical Software, 33(6), 1-48. URL
%   http://www.jstatsoft.org/v33/i06/.


\end{document}




















	%
% 	\subsection{Empirical Control Limits}
% 	tell the story of our logic exploring the empirical approach
% 		\begin{itemize}
% 		\item how to best get Control Limits.
% 		\item idea: since the data are generated from MCMC samples why not 
% 		\end{itemize}

% 		\begin{itemize}
% 		\item Basic SPC
% 		\item EWMA
% 		\item model maximum $\EIx$ as a log-Normal
% 		\item since 
% 			\begin{itemize}
% 			\item 
% 			\end{itemize}
% 		\end{itemize}
		
% 		\subsubsection{Implementation}
% 		\begin{itemize}
% 		\item Normal assumption on $\EIx$ (see Rosebrock for example)
% 		\item 0 bound, model the log.
% 		\item applied modeling skeme 
% 			\begin{itemize}
% 			\item sliding window
% 			\item most recent bata looking back in time.
% 			\item tuning parameter $w$
% 			\item behaviour of $w$ (for choosing $w$)
% 			\item code apendix, using qcc
% 			\end{itemize}
% 		\item convergence rule
% 			\begin{itemize}
% 			\item control in the window
% 			\item out of control outside of the window (this means that you have moved)
% 			\end{itemize}
% 		\end{itemize}


% 	\begin{itemize}
% % 	\item Optimization: Background and our Philisophy
% % 		\begin{itemize}
% % 		\item gradient free \cite{noGradBook}
% % 		\item GA({\color{red}cite \cite{noGradBook} }), Simulated Annealing({\color{red}cite} \cite{noGradBook} ), Pattern Search({\color{red}cite} \cite{noGradBook} ), Trust Regions({\color{red}cite} \cite{noGradBook} ) 
% % 		\item benefits of a model based approach (ie. uncertainy measures)({\color{red}cite})
% % 		\item uncertainty measures as convergence criteria
% % 		\end{itemize}
% 	\item Gaussian Process Surrogate Model(cite), thus E[I]({\color{red}cite})
% 		\begin{itemize}
% 		\item use as an optimization tool (surrogate model)
% 		\end{itemize}
% % 	\item Statistical process control fr monitoring convergence
% % 		\begin{itemize}
% % 		\item General \cite{shewhartBook} (ie. no EWMA)
% % 		\end{itemize}
% 	\end{itemize}

% \section{Treed Gaussian Process Surrogate Model}
% 	\begin{itemize}
% 	\item how does a Gaussian process work({\color{red}cite})
% 	\item Treed partioning for added flexability({\color{red}cite})
% 	\item flexability and advantage of such a surrogate model({\color{red}cite?})
% 	\end{itemize}

% \section{Uncertainty measures associated with Gaussian Process}
% 	\begin{itemize}
% 	\item mention some measures
% 	\item Choose $\EI$, why? \cite{gBook}, \cite{tgp2}
% 	\item some characteristics [bounded at 0]
% 	\item which $\E{~\text{I}^g(\bm{x})~}$, ie. which $g$? 
% 	\end{itemize}

% \section{Optimization Scheme}
% 	\begin{itemize}
% 	\item Optimization Proceedure \cite{tgp2}
% 		\begin{itemize}
% 		\item code appendix, using tgp
% 		\end{itemize}
% 	\item advantages of model based approach for convergence sake
% 	\item $\EIx$ Behaviour for convergence
% 	\item taste of SPC
% 	\end{itemize}

% \section{Statistical Process Control}
% 	\subsection{Shewhart's $\bar x$ chart}
% 	\begin{itemize}
% 	\item the notion of control. \cite{shewhartBook}
% 	\item how they work, and philosphy.
% 	\item stumbling blocks for convergence for me.
% 		\begin{itemize}
% 		\item {\color{red} list them}
% 		\end{itemize}
% 	\end{itemize}
% 	
% 	\subsection{Exponentially Weighted Moving Average Chart}
% 		\subsubsection{General Skeme}
% 		\begin{itemize}
% 		\item EWMA philosophy.
% 		\item Tracking slight changes and why thats better for this problem.
% 		\item How it works ({\color{red}derivation cite}).
% 		\item basic statistics
% 		\end{itemize}
