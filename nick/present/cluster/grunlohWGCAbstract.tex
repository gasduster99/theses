In order to effectively manage exploited populations, accurate estimates
of commercial fisheries catches are necessary to inform monitoring and
assessment efforts. In California, the high degree of heterogeneity in
the species composition of many groundfish fisheries, particularly those
targeting rockfish (\emph{Sebastes}), leads to challenges in sampling
all market categories, or species, adequately. Limited resources and
increasingly complex stratification of the sampling system inevitably
leads to gaps in sample data. In the presence of sampling gaps, ad-hoc
point estimation is currently obtained according to historically derived
``data borrowing'' protocols which do not allow for tractable
uncertainty estimation. In order to move from the current, but
admittedly rigid sampling design, we have continued previous exploratory
efforts to develop, and apply, Bayesian hierarchical models of the
landing data to estimate species compositions. Furthermore, we introduce
a formalized method for discovering consistent ``borrowing'' strategies
across overstratified data. Our results indicate that this approach is
likely to be more robust than the current system, particularly in the
face of sparse sampling. Additionally, our method should also help
inform, and prioritize, future sampling efforts. Perhaps more
significantly, this approach provides estimates of uncertainty around
species-specific catch estimates.
