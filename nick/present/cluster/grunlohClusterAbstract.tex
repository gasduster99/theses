\documentclass[12pt]{article}

%
\usepackage[margin=1in]{geometry}
\usepackage{authblk}

%
\author[1]{Nick Grunloh}
\author[2]{E.J. Dick}
\author[1]{Herbie Lee}

%
\affil[1]{\small Department of Statistical Science, Baskin School of Engineering, University of California Santa Cruz, 1156 High Street, Santa Cruz, CA 95064, USA}
\affil[2]{\small Fisheries Ecology Division, Southwest Fisheries Science Center, National Marine Fisheries Service, NOAA, 110 McAllister Way, Santa Cruz, CA 95060, USA}

%
\date{\today}

%
\title{A Metamodel Based Clustering of Fisheries Model Parameter Estimation Behavior}

%
\begin{document}

%
\pagenumbering{gobble}

%
\maketitle
%
\begin{abstract}
Integrated fisheries models are based upon differential equations which model 
stock dynamics through time. Fisheries are largely managed based upon 
quantities derived from the equilibrium equations of these dynamics, known as 
Reference Points (RP). RP behavior is primarily driven by the functional form 
of the productivity assumed in the differential equations. Mangel et. al. 
(2013) demonstrate that the most commonly used models of productivity limit the 
domain of RPs due to a lack of flexibility induced by their two-parameter 
functional forms. Three-parameter models of production release this theoretical 
RP limitation (Punt \& Cope, 2019). Nonetheless, two-parameter models of 
productivity are overwhelmingly used in practice. When RP model misspecification 
of this type is present in population dynamics models, what are the useful limits 
of statistical inference with respect to estimating these RPs? Here, a simulation 
environment is designed which explores how misspecified two-parameter production 
models bias RP inference. Using a Gaussian Process metamodel of the inferred RPs 
(under two-parameter productivity), the full theoretical space of RP bias 
behavior is explored. This structured simulation setting allows for clustering 
of RP failure modes which use the metamodel to predict when a given species is 
most likely to be subject to catastrophic model failure. 
\end{abstract}

%
\end{document}
