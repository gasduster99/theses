%%
%z = FF+M
%kr = k*W/w
%%
%one = ( (z*(z+k))/(alpha*w*(z+kr)) )^gamma
%two = (gamma*FF/alpha/w) * ((z*(z+k))/(alpha*w*(z+kr)))^(gamma-1)
%thr = kr*(k-kr)/((z+kr)^2)
%
\newcommand{\kr}{ \frac{\kappa w_\infty}{w(a_s)} }
\newcommand{\one}{
        \left(\frac{Z(Z+\kappa)}{\alpha w(a_s)(Z+\kr)}\right)^\gamma
}
\newcommand{\two}{
        \left(\frac{\gamma F}{\alpha w(a_s)}\right) \left(\frac{Z(Z+\kappa)}{\alpha w(a_s)(Z+\kr)}\right)^{\gamma-1}
}
\newcommand{\thr}{
        \frac{\left(\kr\right)\left(\kappa-\kr\right)}{(Z+\kr)^2}
}
%
\newcommand{\oneA}{
        \left(\frac{Z^*(Z^*+\kappa)}{w(a_s)(Z^*+\kr)}\right)^\gamma
}
\newcommand{\twoA}{
        \left(\frac{\gamma F^*}{w(a_s)}\right) \left(\frac{Z^*(Z^*+\kappa)}{w(a_s)(Z^*+\kr)}\right)^{\gamma-1}
}
\newcommand{\thrA}{
        \frac{\left(\kr\right)\left(\kappa-\kr\right)}{(Z^*+\kr)^2}
}



%
\section{Introduction}

%
\begin{itemize}
\item the delay model: \cite{schnute_general_1985} \cite{schnute_general_1987} \cite{fournier_length-based_1987}.
\item discrete: \cite[pg. 334]{hilborn_quantitative_1992}
\item \cite{walters_continuous_2020}
\item automatic accounting for cohort cycles
\end{itemize}

%
\section{Methods}

%
\subsection{Delay Differential Model}

%
\begin{wrapfigure}{r}{0.45\textwidth} %[17]{r}[0pt]{0pt}%
%\begin{figure}[h!]
\vspace{-1cm}
%\includegraphics[width=0.5\textwidth]{plots/derisoSrr.png}
%\begin{minipage}[h!]{0.64\textwidth}
\includegraphics[width=0.49\textwidth]{plots/vbOpt.png}
%\end{minipage}
%\begin{minipage}[h!]{0.3\textwidth}
\vspace{-1cm}
%\hspace*{-1cm}
\caption{
%\onehalfspacing
The typical composition of allometric weight ($b=3$) with VB growth in length, as
approximated by VB growth in weight directly.
%A comparison of 
%with the typical assumption of VB growth in length. 
}
\label{SrrPT}
%\end{minipage}
\end{wrapfigure}

%
Age structured fisheries models typically assume %Von Bertalanffy (VB) growth 
\cite[VB]{von_bertalanffy_quantitative_1938} gorwth in length with age. To model
weight the assumption of VB growth in length is composed with a power law 
relating length to weight, $w=al^b$. 
%The statistical model then assumes observed indicies of abundance are proportional to weight. 
%
Since $b$ is usually $\sim3$ this composition of assumed functional forms
typically results in a monotonically increasing sigmoidal curve of weight with age.
When $b\le1$ weight at age takes a VB-like form with $b=1$ resulting in
an exact correspondence of simulanious VB-growth in length and weight.

%
The delay model slightly abridges these relationships by directly assuming VB
growth in weight as follows,
%%
%\begin{align}
%\frac{dw}{da} &= \kappa(w_\infty-w(a)) \label{wODE}
%\end{align}
%
\begin{align}
w(a) &= w_\infty(1-e^{-\kappa (a-a_0)}). \label{vbGrowth}
\end{align}
%
$\kappa$ is a parameter that controls the instantaneous rate of individual
growth (in weight) with age. $w_\infty$ is the maximum weight of individuals
in the population, and $w(a)$ is the average weight of an individual at
age $a$. The parameter $a_0$ controls the age at which individuals are assummed
to have zero weight; by letting $a_0<0$ this allows fish of age zero to have
positive weight. Rather than taking a sigmoidally increasing function, VB growth
directly in weight results in an monotonically inceasing curve that asymptotes
with a strictly decreasing growth rate with age.
{\color{red}(only a good approximation for older ages where growth begins to decline)}

%
Together with VB growth, the delay model is derived from the assumption that
both natural mortality and fishing selectivity are separately propotional
%the total mortality rate (from both natural and fishing mortality) is proportional 
to a common heavyside step function with age. That is to say, before a threshold
age of selectivity, $a_s$, the population is assumed not to experience any
mortality whatsoever, but all fish older then $a_s$ experience the same rate
of natural mortaility. Simulaneously all fish older than $a_s$ are equally
vulnerable to fishing (i.e. knife edge selectivity at age $a_s$), although
fishing effort may vary from through time.

%
\cite{walters_continuous_2020} shows that within these assumptions the
following delay differential system of equations exactly models the population
dynamics of the total exploitable biomass $B(t)$ and number of indivuduals $N(t)$
through time.
%%
%\begin{align}
%B(t) = \int^\infty_{a_s} N(a, t)w(a) da
%\end{align}
%
\begin{align}%\kappa->0 slower than a0->\infty
%B(a, t) &= w(a, t)N(a, t)\\ 
%\frac{dB}{dt} &= \overbrace{w(k)R(B(t-k))}^\text{Recruitment Biomass} + \overbrace{\mu w_\infty N(t)}^\text{Growth} - \overbrace{(M+F(t)+\mu)B(t)}^\text{Biomass Loss}\\
&\frac{dB}{dt} = w(a_s)R(B;\theta) + \kappa \left[w_\infty N-B\right] - (M+F)B \label{bEq}\\
&\frac{dN}{dt} = R(B;\theta) - (M+F)N \label{nEq}
%&R(B;[\alpha, \beta, \gamma]) = \alpha B(t-a_s)(1-\beta\gamma B(t-a_s))^{\frac{1}{\gamma}} \label{srr}\\
%&w(a) = w_\infty(1-e^{-\kappa a}) \label{vbGrowth}
\end{align}

%
This formulation separates the number of individuals in the population from the
biomass of the population. The dynamics of $N$, as seen in Eq (\ref{nEq}), are
very similar to that of the {\color{green}production models previously presented}, 
however the role of the production function is now filled by a "recruitment"
function, $R(B)$, which describes the number of new individuals recruiting into the
expoitable population as a function of exploitable biomass. In turn, the biomass
dynamics are coupled to the numbers dynamics by the assumption of VB growth with
growth parameters appearing in Eq (\ref{bEq}), converting population numbers
into biomass and accounting for the growth of biomass with age.

%
Eq (\ref{bEq}) of the above model expands the notion of biomass production into the
processes of recruitment, individual growth, and maturity. The term $w(a_s)R(B;\theta)$
represents the biomass of new recruits; with $w(a_s)$ representing the weight of individuals
at the age of maturity, $a_s$, and $R(B;\theta)$ representing the number of new recruits
entering the exploitable population at time $t$. The negative term, $(M+F)B$, represents all
causes of mortality as it is applied to biomass. Finally, the term $\kappa \left[w_\infty N-B\right]$
accounts for the net growth of the existing biomass by discounting the limiting maximal individual
growth rate by metabolic weight loss proportional to $B(t)$. This term, together with the delay
structure in $R$, provides the major computational savings of the delay differential setting, as
compared with full age structured models, by automatically keeping track of changes in the mean
size and growth associated with changes in recruitment as cohorts mature into the population.
%The framework is likely to perform very well at this task, considering that growth and natural 
%survival rates tend to be fairly stable over time in fishes.

%
Often a BH functional form is assumed for the stock recruitment relationship, but any adequatly
flexible family of functions may model this relationship. For the sake of evaluating the adequacy
of assumed BH recruitment the simulation setting below is derived for the delay model under the
assumption of the generalized three parameter Schnute recruitment as follows.
%
\begin{align}
R(B;[\alpha, \beta, \gamma]') = \alpha B(t-a_s)(1-\beta\gamma B(t-a_s))^{\frac{1}{\gamma}} \label{srr}
\end{align}
%
The parameters $\bm{\theta}'=[\alpha, \beta, \gamma]$ %\alpha$, $\beta$, and $\gamma$, 
function similarly in this setting as previously described in Section (\ref{}).
That said, since the delay model explicitly parses out growth in it's dynamics,
these parameters only describe the net processes of larval production, and maturation
into the population, where as the production model used these parameters to
also model the net effects of growth on biomass production. %NOTE: production model includes net growth in in the nonlinear form of P, while the delay model used VB Growth params (functional form) on top of this nonlinearity. 
The $\gamma$ parameter generalizes the family to model varying degrees of
decreasing recruitment for large biomasses as $\gamma$ increases. The Schnute
function is exactly equivalent to BH recruitment at the special case when
$\gamma=-1$, it passes through the Ricker model as $\gamma\rightarrow0$, and
Logistic recruitment occurs when $\gamma=1$.

%%the exploitable biomass of the population becomes large
% controls the behavior of recruitment from the special case of BH recruitment at $\gamma=-1$, 
%The structure of the Schnute function here is similar to that of the previously described, with $\gamma$ . 
%
Since the delay model assumes knife edge selectivity, at age $a_s$, the term
$B(t-a_s)$ appears in $R$. That is to say fish recruiting into the exploitable
population are the result of larval production of biomass $a_s$ time
units in the past. This is because fishing selectivity is only assumed to occur
for fish that are at least $a_s$ time units old and thus fish younger than $a_s$
are not exploitable. This waiting period requires that new recruits be the
result of spawning biomass $a_s$ time units in the past. Modeling maturity in
this way results in dynamics equations which are a system of delay differential
equations as opposed to the simple ODEs that arrise in the production model
setting.
%which is where the delay differential equation  

%
\begin{itemize}
        %\item parameters $\alpha$, $\beta$, and $\gamma$,
        %\item The BH and Logistic production functions arise when $\gamma$ is fixed to -1 or 1 respectively. 
        %\item The Ricker model is a limiting case as $\gamma\rightarrow0$. %\shortcite{schnute_general_1985}.
        %\item For $\gamma<-1$ a family of strictly increasing Cushing-like curves arise,
        %       culminating in linear production as $\gamma\to-\infty$. These special cases form
        %       natural regimes of similarly behaving production functions as seen in Figure (\ref{sRegimes}).
        %\item time delay
        \item[$\sim$] interpretation of recruitment (larval production, recruitment) [growth external] vs. production (larval production, recruitment, growth)
\end{itemize}

\begin{itemize}
\item general structure: \cite{walters_continuous_2020} \cite[pg. 334]{hilborn_quantitative_1992}
\item growth: \cite{von_bertalanffy_quantitative_1938}
\item recruitment: \cite{schnute_general_1985, schnute_analytical_1998}
\end{itemize}


%
\subsection{Reference Points}

%
Deriving reference points for the delay model under Schnute recruitment is
conceptually similar to the production model setting. The additional nonlinear
VB growth assumptions along side Schnute recruitment quickly make the
expressions look somewhat unweildy, although analytical solutions can still be
derived for most of the same quantities (although complicated by growth parameters).

%
Starting from Eqs. (\ref{bEq}) and (\ref{nEq}), setting both $\frac{dB}{dt}$
and $\frac{dN}{dt}$ simultaneously equal to zero, and solving for $B$ and $N$
as a function of fishing, gives the equilibrium biomass and numbers equations.
%
\begin{align}
\bar{B}(F) &= \frac{1}{\beta\gamma} \left( 1 - \Big(\frac{(F + M) (F + M + \kappa)}{\alpha w(a_s)(F + M + \kr)}\Big)^\gamma\right) \label{BF}\\
%\end{align}
%\begin{align}
\bar{N}(F) &= \frac{\alpha\bar{B}(F)(1-\beta\gamma\bar{B}(F))^{1/\gamma}}{F+M} \label{NF}
\end{align}
%
Eq. (\ref{NF}) is just $\frac{R(\bar{B})}{F+M}$, and is coupled to $\bar{B}(F)$
where most of the dynamics appear. Eq. (\ref{BF}) resembles Eq (\ref{BsEq})
from the simple production model setting although the growth parameters
$\kappa$, $w_\infty$ and $w(a_s)$, make slight adjustments to the balance of the
maximum rate of recruitment and mortaility rate to give an expression for
equilibrium biomass that accounts for the factors of individual growth.
% for the produce to a more complex expression for equilibirum biomass.

%
Expressions for $B_0$ and $B^*$ are attained by evaluating $\bar{B}(F)$ at
$F=0$ and $F=F^*$ respectively. Calculation of $F^*$ typically involves %amounts to %requires %Obtaining an expression for 
maximization of equilibrium yield, \mbox{$\bar{Y} = F\bar{B}(F)$.} While it was not
possible to analytically maximize $\bar{Y}$, stable numerical solutions for
calculating $F^*$ were obtained by numerically solving for the roots of the
analytical derivative of equilibrium yield with respect to $F$. Below a greatly
simplifed expression for $\frac{d \bar{Y}}{dF}$ is shown; the substitution
$Z=F+M$ (total mortality rate) has been made to produce a more compact expression.

%
\vspace{-0.75cm}
\begingroup
\scriptsize
\begin{align}
\frac{d \bar{Y}}{dF} &= \frac{1}{\beta\gamma}\left[ 1 - \one - \two \left( 1 + \thr \right) \right]\label{dBdFS}
%&= (1-\left(\frac{(F+M)*(F+M+\kappa)}{\alpha*(F*w(a_s)+M*w(a_s)+\kappa*w_infty)}\right)^\gamma-(F*(((F+M)*(F+M+\kappa))/(\alpha*(F*w(a_s)+M*w(a_s)+\kappa*w_infty)))^(\gamma-1)*\gamma*(\alpha*(F*w(a_s)+M*w(a_s)+\kappa*w_infty)*(2(F+M)+\kappa)-(F+M)*(F+M+\kappa)*\alpha*w(a_s)))/(\alpha*(F*w(a_s)+M*w(a_s)+k*w_infty))^2)/(\beta\gamma) \label{dBdFS}.
%&= ( 1-\Big(\frac{(F+M)(F+M+\kappa)}{\alpha w(a_s)(F+M+\kappa w_\infty/w(a_s))}\Big)^\gamma - ( F*( ((F+M)*(F+M+\kappa))/(\alpha*(F*w(a_s)+M*w(a_s)+\kappa*w_infty)) )^(\gamma-1)*\gamma*(\alpha*(F*w(a_s)+M*w(a_s)+\kappa*w_infty)*(2(F+M)+\kappa)-(F+M)*(F+M+\kappa)*\alpha*w(a_s)) )/( \alpha*(F*w(a_s)+M*w(a_s)+k*w_infty))^2)/(\beta\gamma) \\
\end{align}
\endgroup
%
$F^*$ is calculated as the numerical root, w.r.t. $F$, of the above expression.
The numerical root is calculated using the base R uniroot function which
employs a derivative free search given by \cite{brent_chapter_1973}. %\shortciteA{brent_chapter_1973}.

%
\subsubsection{BH Constraint}

%
\begin{wrapfigure}{r}{0.50\textwidth} %[17]{r}[0pt]{0pt}%
%\begin{figure}[h!]
\vspace{-2.75cm}
%\includegraphics[width=0.5\textwidth]{plots/derisoSrr.png}
%\begin{minipage}[h!]{0.64\textwidth}
\includegraphics[width=0.54\textwidth]{../ddBias/rpSpaceww.png}
%\end{minipage}
%\begin{minipage}[h!]{0.3\textwidth}
\vspace{-1.5cm}
%\hspace*{-1cm}
\caption{
%\onehalfspacing
The space of BH RPs for the delay model as a function of $\kappa$ and $a_s$.
The RP space is plotted for $80\times80$ combinations of $\kappa\in[0.1, 2]$
and $a_s\in[0.1, 10]$. The color drawn is the resulting value of $w(a_s)$
mapped between blue and red.
%the result of mapping $\kappa$ and $a_s$ values to the red and blue components of the RGB color model repsectively (with G=0). 
$\frac{1}{x+2}$ is plotted in black for reference.
%
}
\label{rpSpace}
%\end{minipage}
\end{wrapfigure}
%\end{figure}

%
In the simple production model the BH constrained RPs are fixed to $\frac{1}{x+2}$.
In the delay differential modeling setting the constrained BH RP set is
complicated by the growth parameters $a_s$ and $\kappa$.
Under BH recruitment these parameters of the delay model slightly %the growth parameters $a_s$ and $\kappa$  
influence this relationship as seen in Figure (\ref{rpSpace}). That said,
the influence of $a_s$ and $\kappa$ on RPs is still largly limited to a
confined region of reference point space which resembles the $\frac{1}{x+2}$
form. In fact the confined region of RPs is bounded above by $\frac{1}{x+2}$. %bounding the region above by $\frac{1}{x+2}$.
In Figure (\ref{rpSpace}) notice that for values of $a_s$ and $\kappa$ that
result in high $w(a_s)$ (high values of $\kappa$ and small values of $a_s$ seen
in red) the BH RP space converges to $\frac{1}{x+2}$ as derived in the simple
production model setting. In opposition to the simple production model limit,
when $w(a_s)$ is low (as seen in the more blue region of Figure(\ref{rpSpace})), RPs
decrease as the influence of growth in the dynamics increases.
%The opposite limit with low values of $\kappa$ and high values $a_s$ (blue region) depresses RPs away from $\frac{1}{x+2}$. 

%
\subsection{Delay Differential Integration}

%
The delay model belongs to a class of differential equations known as delay
differential equations (DDE). The delay arrises from the $B(t-a_s)$ terms
found in the recruitment function. Solving DDEs require special care which
depends on the nature of the time delay. The addition of time-varying delays,
many different delays, or very small delays (delays below the step size of the
numerical integrator) results in some of the more challenging settings for
solving DDEs. However with a single stationary model of the age of selectivity,
the delay model in this setting represents one of the most straight forward
DDE structures. The most numerically challenging case presented here arrises
in the case of the limiting production model when $a_s\to0$ while $\kappa\to\infty$.
That said the limiting production model can be approximated for values of
$a_s\approx0.1$, and it was straightforward to ensure that the step size of
the integrator remained reasonably below 0.1.

%
The DDE presented here is integrated with the initial values fixed at $B_0$
and $N_0$ as given by Eqs. (\ref{BF}) and (\ref{NF}) with $F=0$ at any given
configuration of $\bm{\theta}$ and growth parameters. %$\kappa$, $w_\infty$ and $a_s$. 
The system given in Eqs. (\ref{bEq}) and (\ref{nEq}) are then solved
numerically using the implicit Livermore Solver (lsode) as implemented in the
\verb|dede| function of the R package \verb|deSolve| \cite{soetaert_solving_2010}.
The \verb|dede| solver provides many methods for integrating DDEs, but lsode
was chosen because it is an implicit method that runs relatively quickly with
a relatively smaller footprint in system memory as compared with other methods.
The radau method was also tried in more computationally challenging settings
with good results (albeit running more slowly that lsode). Ultimatly the
simulated parameter space did not produce DDEs that require the more expensive
radau integrator to solve accurately.

%delays are on the order of the step size (vanishing delays) are difficult to solve. (dde CRAN)
%
%https://cran.r-project.org/web/packages/dde/vignettes/dde.html

%
\subsection{Simulation Design}

%
Similarly as previously described in Section (\ref{sSim}) the relationship
between RPs $\mapsto$ $\theta$ cannot be fully expressed analytically for the
Schnute delay model. However, just as in the production model setting,
simulation only requires enough knowledge of these mappings to gather a list
of $(\alpha, \beta, \gamma)$ tuples and the corresponding RPs in some reasonable
space-filling design over RP space. %for the purposes of simulation 

%Like the Schnute production model setting, 
In the delay model a partial mapping for
$\big(F^*, B_0\big) \mapsto \big(\alpha(\cdot, \gamma), ~\beta(\cdot, \cdot, \gamma)\big)$
%in the delay modelling setting 
can be derived analytically in terms of RPs and $\gamma$. The substitution
$Z^*=F^*+M$ is made where $F^*$ and $M$ appear together to produce a more
compact expression.

%
\begingroup
\scriptsize
\begin{align}
\alpha & = \left[ \oneA + \twoA \left( 1 + \thrA \right) \right]^{\frac{1}{\gamma}} \label{aDelay}\\
\beta &= \frac{1}{\gamma B_0} \left( 1 - \Big(\frac{M (M + \kappa)}{\alpha w(a_s)(M + \kr)}\Big)^\gamma\right) \label{bDelay}%\\
%\frac{B^*}{B_0} &= \frac{ 1 - \Big(\frac{(F^* + M) (F^* + M + \kappa)}{\alpha w(a_s)(F^* + M + \kr)}\Big)^\gamma }{ 1 - \Big(\frac{M (M + \kappa)}{\alpha w(a_s)(M + \kr)}\Big)^\gamma }
\end{align}
\endgroup

%
Above Eq. (\ref{aDelay}) results from setting \mbox{Eq. (\ref{dBdFS})} equal to zero
and solving for $\alpha$, and \mbox{Eq. (\ref{bDelay})} results from solving the
$\bar{B}(0)$ expression, as derived from \mbox{Eq. (\ref{BF}),} for $\beta$. The system
is completed by further working with the $\frac{\bar{B}(F^*)}{\bar{B}(0)}$ expression,
as seen below, to identify $\gamma$.
\begin{align}
\frac{B^*}{B_0} &= \frac{ 1 - \Big(\frac{(F^* + M) (F^* + M + \kappa)}{\alpha w(a_s)(F^* + M + \kr)}\Big)^\gamma }{ 1 - \Big(\frac{M (M + \kappa)}{\alpha w(a_s)(M + \kr)}\Big)^\gamma }
\label{gDelay}
\end{align}

%
The system formed by collecting Eqs. (\ref{aDelay}), (\ref{bDelay}), and
(\ref{gDelay}) can be navigated similarly to Eq. (\ref{abgSys}) in the Schnute
production model setting. For a population experiencing natural mortality $M$,
VB growth with paramters $\kappa$ and $w_\infty$, and age of selectivity $a_s$
the above system can fully specify $\alpha$ and $\beta$ for a given $\gamma$,
by fixing $F^*$, $B_0$, and $\frac{B^*}{B_0}$. For a given $\gamma$ a cascade
of closed form solutions for $\alpha$ and $\beta$ can be obtained, just as in
Section (\ref{sSim}).
%
First $\alpha(\gamma)$ can be computed, and then $\beta(\alpha(\gamma), \gamma)$
can be computed. If $\alpha(\gamma)$ is filled back into the expression for
$\frac{B^*}{B_0}$, the system collapses into a single onerous expression for
$\frac{B^*}{B_0}(\alpha(\gamma), \gamma)$.For brevity, define the function
\mbox{$\zeta(\gamma)=\frac{B^*}{B_0}\big(\alpha(\gamma), \gamma, F^*, M\big)$}
based on Eq. (\ref{gDelay}).

%
Again rather than inverting $\zeta(\gamma)$ for $\gamma$, $\gamma$ is the
sampled so that the overall simulation design is space filling as described in
Section (\ref{sLHS}). Given the sampled $\gamma$, the cascade of $\alpha(\gamma)$, 
and then $\beta(\alpha(\gamma), \gamma)$, can be computed, and the Schnute
delay model is fully defined by a given $(\frac{F^*}{M}, \frac{B^*}{B_0})$.
%
While conceputally this framing is similar to the Schnute production model,
the analytical expressions are more complex, and numerically trecherous, since
growth parameters appear explicitly here. Other ways of navigating the RPs $\mapsto$ $\theta$
system are possible, but for the sake of numerical stability this strategy has
proven the most reliably accurate by limiting exposure to numerical error propogation.

%
Each design location defines a complete Schnute delay differential model with
the given RP values. Indices of abundance are simulated from the Schnute model
at each design location, a small amount of residual variation, $\sigma = 0.01$,
is added to the simulated index, and the data are then fit with a misspecified
BH model. The design captures various degrees of model misspecification
relative to the BH model, so as to observe the effect of recruitment
misspecification upon RP inference.

%
{\color{red}
point to catch, and LHS design, and Metamodel.
}

%
\subsection{Parameter Estimation}
%
\begin{itemize}
        %\item layout likelihood setup
        %\item mention recreational data numbers likelihood
        %\item point out similarity of B/N solutions
        \item I use B only here
        \item quick statement of inference, and reference to previous section
\end{itemize}

%
Let $I_t$, $t\in\{1,2,3,...,T\}$, be a series of indicies of abundance,
proportional to biomass, as simulated from the Schnute Delay model. These data
are modelled with the following log-normal observation model that has been
intentionally constrained to BH recruitment,
%is then intentially constrainted to BH recruitment and a log-normal observation model as follows, % by fixing $\gamma=-1$.
%Growth parameters: $\bm{\phi}'=[\kappa, w_\infty, a_0, a_s]$
%
%The observation model for the fitted model is log-normal such that, 
%|q, \sigma^2, \bm{\theta}, \bm{\phi}
\begin{align}
I_t \sim LN(q B_t(\bm{\theta}, \bm{\phi}), \sigma^2). \label{bL}
\end{align}
%
$B_t(\bm{\theta}, \bm{\phi})$ is the biomass solution of the BH constrained DDE system. %of DDE defined in Eq. (\ref{bEq}). 
The BH constraint isimplemented by fixing $\gamma=-1$ so that $\bm{\theta}'=[\alpha, \beta, \gamma=-1]$.
$\bm{\phi}$ is a vector of growth and maturity parameters,
$\bm{\phi}'=[\kappa, w_\infty, a_0, a_s]$. The nuisance parameter $q$ models
the proportionality constant of the index with process biomass, and $\sigma^2$
models residual variation of the index.

%
In this setting, $\bm{\phi}$ and $q$ are fixed %at the true value of 0.0005 
to focus on the inferential affects of model misspecification on recruitment
parameters and RPs. Without an explicite mechanism for the delay model to incorporate
age data, under the BH model $\bm{\phi}$ is not well informed and would
tyically be estimated externally for data limted stocks. Under BH recruitment
$\phi$ can only slightly impact RPs as seen in Figure (\ref{rpSpace}).
%the BH model only contributes largly unimportant for the mapping of RPs since  
%{\color{red} $\bm{\phi}$ tyically estimated externally; 
%the components of $\phi$ only slightly impact RPs as seen in Figure (\ref{rpSpace})}.

%
$\sigma^2$ and $\theta$ are reparameterized to the log scale and fit via MLE.
Reparameterizing the parameters to the log scale improves the reliability of
optimization, in addition to facilitating the use of Hessian information for
estimating MLE standard errors. Given that the biological parameters enter the
likelihood via a nonlinear differential equation, and further the parameters
themselves are related to each other nonlinearly, the likelihood function can
often be difficult to optimize. A hybrid optimization scheme is used to
maximize the log likelihood to ensure that a global MLE solution is found. The
R package GA \cite{scrucca_ga_2013, scrucca_extensions_2017} is used to
run a genetic algorithm to explore parameter space globally. Optimization
periodically jumps into the L-BFGS-B local optimizer to refine optima within a
local mode. The scheme functions by searching globally, with the genetic
algorithm, across many initial values for starting the local gradient-based
optimizer. The genetic algorithm serves to iteratively improve hot starts for
the local gradient-based optimizer. Additionally, optimization is only
considered to be converged when the optimum results in an invertible Hessian at
the found MLE.

\begin{itemize}
%$\bm{\phi}$ fixed, $q$ fixed,\\ 
%MLE $\bm{\theta}$ and $\sigma^2$\\
\item fixed $M=0.2$, $a_0=-1$, $w_\infty=1$
\item play with $\kappa$ and age of selectivity $a_s$
%Look at values above and adapt from schnute paper.
\end{itemize}

%
\subsubsection{Numbers Indicies}

%
While not utilized here, age structured models may commonly model
%the delay model is also capable of modeling 
indicies as proportional to numbers rather than (or simultaiously to)
biomass. When solving the DDE, Eq. (\ref{nEq}) points out that the full DDE
solution will expose a numbers solution simultaneously with a biomass solution
that may be used for these purposes. These solutions are often quite similar
since the main driver of process behavior comes from the form of $R$ which is
shared among $N$ and $B$.
However, it is common on the west coast of the US that indicies derived from commercial
fisheries are measured as weights %biomass Index observations
while indicies derived from recreational fisheries are often measured as counts.
%represent indicies that are better modeled as proportional tocounts rather than biomass. 
If a numbers index, $J_t$, is observed alongside the previously
mentioned biomass index, the following likelihood component is often added as a
conditionally independent component of the likelihood, %as follows,
%|p, \tau^{2}, \bm{\theta}, \bm{\phi}
\begin{align}
J_t \sim LN(p N_t(\bm{\theta}, \bm{\phi}), \tau^{2}) \label{nL}.
\end{align}
%
$N_t(\bm{\theta}, \bm{\phi})$ is the numbers solution of the DDE system.
$\bm{\theta}$ and $\bm{\phi}$ are the productivity and growth parameters shared in 
common with the biomass component. $p$ and $\tau^2$ are then the analogous
proportionality constant and residual variation of the numbers index respectively.

%
\subsection{GP Metamodel}

%
{\color{red}
point to catch, and LHS design, and Metamodel.
}

%
\subsection{Clustering Model Failure}

%
Considering the behavior observed in Section (\ref{schnuteLowResults}), where
$\frac{F_{MSY}}{M}$ is dramatically underestimated, it is natural to ask
where specifically in RP space we might see this catastrophic failure of the
BH model.
%
The structure of RPs under the BH model suggests several natural avenues for
forming hypotheses to identify highly misspecified RP regions. The single
clearest feature to identify are cases where $\frac{F_{MSY}}{M}$ is
heavily under-estimated. Here this idea is expressed by a hypothesis testing
inspired framework that uses the GP metamodel to proprogate estimate
uncertainty across the simulated space of misspecified BH RPs.
%as a surrogate for the distribution of 
%$\frac{F_{MSY}}{M}$ across degrees of misspecified BH models. 
This allows for a rejection threshold (against the null hypothesis that BH RP
estimates are unbiased) to be derived in terms of the GP predictive structures
to define a classifier for identifying where BH inference breaks down
broadly over RP space.


%
\clearpage
%
Recall that the metamodel models MLE estimates of $log(F_{MSY})$ under the misspecified BH model.
%is a metamodeled quantity, 
%corresponding to $$, $log(F_{MSY})$, corresponding to is the metamodeled 
Thus, for a given set of RPs, $\textbf{x}$, of the BH metamodeled
quantity is given by kriging prediction as $N(\hat y(\textbf{x}), \hat \sigma^2(\textbf{x}))$,
where $\hat y(\textbf{x})$ is the kriging mean (as previously described in
Eq. (\ref{gpYHat})) and $\hat \sigma^2(\textbf{x})$ provides estimate
uncertainty via the kriging predictive variance given by,
\begin{equation} %k(x) a n−vector with kν,j (x) = K(x, xj ), for all xj ∈ X
       \hat \sigma^2(\textbf{x}) = \textbf{R(x, x)} - \textbf{r(x)}'\bm{R}^{-1}_{\bm{\ell}}\textbf{r(x)}.
\end{equation}

%
Model failure with respect to estimating $\frac{F_{MSY}}{M}$ under the BH
model is measured by the percent error as previously described in Section (\ref{highConRes}).
When the BH model estimates $\frac{F_{MSY}}{M}$ well the percent error is
expected to be small in the following sense,
%In this setting to define the degree of model failure the percent error of the 
%estimate is capped at no more than $P$ 
%with respect to $\frac{F_{MSY}}{M}$ 
%the percent error is capped to 

%
\begin{equation}
\frac{\frac{F_{MSY}}{M}-\frac{\hat{F}_{MSY}}{M}}{\frac{F_{MSY}}{M}}<P. \label{pError}
\end{equation}

%
$P$ defines the extent of model failure on the scale of percent error. For
measuring catestrophic model failure $P$ was chosen to be $0.5$, but smaller values
of $P$ may be chosen to emphasize regions of more subtle model failure. %subtle regions 
%
Thus when the percent error is statistically greater than $P$ the notion that
the BH model estimates $\frac{F_{MSY}}{M}$ well (in $P$-sense) is rejected.

%
For statistical evaluation, it is convienient to rearrange Eq. (\ref{pError})
as \mbox{$\hat{F}_{MSY}>(1-P)F_{MSY}$}. $\hat{F}_{MSY}$ is then distributed as $LN(\hat y(\textbf{x}), \hat \sigma^2(\textbf{x}))$,
and the rejection region is then defined as the RP's for which the $5^{th}$ percentile
from the Log-normal distribution falls below \mbox{$(1-P)F_{MSY}$.}

%
%

%
\clearpage
\section{Results}

%\begin{figure}[h!]
\begin{wrapfigure}{r}{0.50\textwidth}
\vspace{-1.5cm}
\includegraphics[width=0.49\textwidth]{../ddBias/vbCurves.png}
\vspace{-1cm}
\caption{Three hypothetical individual-growth curves, showing $w(a_s)$ on each curve.}\label{vbCurves}
\end{wrapfigure}
%
Figure (\ref{vbCurves}) shows three hypothetical individual-growth/maturity curves 
that span a wide range of RPs. As seen in Figure (\ref{rpSpace}), the larger values of
$w(a_s)$ corresond to less dramatic growth with the red curve demonstrating the
simle (no growth) production model limit ($a_s\rightarrow0$ and $\kappa\rightarrow\infty$).
The cases with smaller $w(a_s)$ values (blue and purple curves) correspond to more dramatic
growth behaviors, with the blue curve where $a_s=2$ and $\kappa=0.1$ representing the most
dramatic growth shown here.
%a more lagged selectivity and slow indivdual-growth.

%
\begin{figure}[h!]
\includegraphics[width=\textwidth]{../ddBias/growthTriptic.png}
\caption{
Biomass dynamics of BH ($left$), Ricker ($center$), and Logistic ($right$)
delay differential models in the low contrast simulation setting. In all cases
$\alpha=1.2$ and $\beta$ is chosen so that each model shares the same
$B_{MSY}$ within each given $\gamma$.
}\label{delayTriptic}
\end{figure}
%
Figure (\ref{delayTriptic}) demonstrates a range of biomass dynamics that the Schnute
delay model can display under a spectrum of growth behaviors with fishing held consistent
at $F_{MSY}$. The three special cases of $\gamma=-1$ (BH), $\gamma\to0$
(Ricker), and $\gamma=1$ (Logistic) recruitment are shown in each of the above
shown growth configurations.
%of growth ranging from the simple (no growth) production model setting when 
%$a_s\rightarrow0$ and $\kappa\rightarrow\infty$, to an opposing case ($a_s=10$ 
%and $\kappa=0.1$) representing extremely lagged selectivity and slow indivdual-growth. 
%By reference with Figure (\ref{rpSpace}), notice that the latter individual-growth/maturity 
%configuration represents a RP set in the deep blue regiem, thus 
%this case represents 
%dynamics where RPs are heavily influenced by individual-growth/maturity, 
%rather than solely determined by the assumed functional form of recruitment as is 

 
%the case with the simple production model. The case $a_s=2$ and $\kappa=0.1$ is 
%shown as an example of dynamics where RPs are more moderatly influenced by 
%individual-growth/maturity. 
%The shortened selectivity lag allows oscillatory individual-growth/maturity 
%dynamics to quickly equilibrate towards a more purely 
%recruitment driven dynamics more quickly than the $a_s=10$ example.
%% right leaning yeild curve
Notice under the most dramatic growth ($a_s=2$ and $\kappa=0.1$) setting, biomass
of the Logistic model comes into equilibrium at $B_{MSY}$ as an oscillating
curve. This effect occures here due to the Logistic model's relatively high $\frac{B^*}{B_0}$
%steep yeild curve for high biomasses 
interacting with the lag in selectivity upon the sudden onset of fishing; this
produces a shock that pushes biomass past $B_{MSY}$ setting up an oscillatory
pattern of recruitment.
%over the steepest regions of the yeild curve. 
One may also observe these oscillations under the Ricker model by exaggerating
the $a_s$ lag as well as the steepness of the Ricker curve. The BH model may
also demonstrate these ocillations, in a heavily lagged setting, by shocking
the population past its relatively low $B_{MSY}$ as a sudden release in fishing applied to a
%over the steepest portion of its yeild curve (a sudden release 
heavily fished population at low equilibrium biomass.

%
\begin{wrapfigure}{r}{0.50\textwidth}
\vspace{-1.5cm}
\includegraphics[width=0.49\textwidth]{../ddBias/rpTriptic.png}
\vspace{-0.75cm}
\caption{Restricted RP-space under each recruitment models,
%The dotted lines show the RP-space of each recruitment model
with each growth curve.}\label{rpTriptic}
\end{wrapfigure}
%
Figure (\ref{rpTriptic}) shows the range of RPs that can be modeled with each
of the BH, Ricker, and Logistic recruitments over the spectrum of
individual-growth/maturitymodels simulated here. Notice that the more dramatic
the growth, the further the RP curve lies from the simple production model,
but each recruitment model reacts differently under each of the given growth
parameters. The Ricker and BH RP-spaces are qualitatively similar in shape
with more dramatic growth settings decreasing $\frac{B_{MSY}}{B_0}$ relative to the
simple production model setting. The Logistic model on the other hand increases
$\frac{B_{MSY}}{B_0}$ relative to the simple production model setting as growth
parameters become more dramatic. It is also worth noting that the Ricker model's
RPs are much less influenced by growth parameters as compared with that of the
BH or Logistic model.

%
\clearpage
%
\subsection{Simple Production Model Limit}

%
Under the delay differential's limiting simple production model ($a_s=0.1$ and $\kappa=10$),
the expectation is that RP inference should be identical to that of the model seen in
Chapter (\ref{schnuteChapter}). By way of verifying this equivalence, Figure (\ref{prodLimit})
demonstrates a virtually identical pattern of RP biases as previously seen in
Figures (\ref{contrastTrio}) and (\ref{bhLowArrows}) (under both of the high and
low contrast settings).

%
\begin{figure}[h!]
\includegraphics[width=0.49\textwidth]{../ddBias/directionalBiasDDSubExpT45N300AS0.1K10.png}
\includegraphics[width=0.49\textwidth]{../ddBias/directionalBiasDDSubFlatT45N150A0-1AS0.1K10N56.png}
\caption{
RP mapping of BH delay model fit to Schnute delay data under the simple (no
growth) production model limit. $left:$ High contrast simulation.
$Right:$ Low contrast simulation.
}\label{prodLimit}
\end{figure}

%\subsubsection{high}
%Just as seen in Section (\label{schnutePics}), here 
%As a limiting case of the delay model in this case, Figure (\ref{prodLimit}) again demonstrates 
%%Just as seen in Chapter (\ref{schnuteChapter}), Figure (\ref{prodLimit}, $left$)
%how RP mapping

%Again under
Indeed in the high contrast setting, Figure (\ref{prodLimit}, $left$) shows
how the BH model induces the same pattern of bias as seen in Chapter
(\ref{schnuteChapter}). There is bias in both RPs (in accordance with the
$\frac{B^*}{\bar B(0)}=\frac{1}{F^*/M+2}$ RP-set) so as to produce a nearly
minimal distance mapping of RPs onto the constrained BH set of RPs.
%\subsubsection{low}
Similarly, in the low contrast setting, Figure (\ref{prodLimit}, $right$) again
shows the same two regiems pattern of RP inference. Firstly, there is a region of
relatively small model misspecification where the minimal distance mapping
is preserved. Secondly, as model misspecification becomes greater (around the
Ricker set) $\frac{F^*}{M}$ begins to be sharply underestimated. Above this
break point in RP estimation inference appears to be driven to the trivial RP
$\frac{F^*}{M}=0$, $\frac{B^*}{\bar B(0)}=0.5$) that is shared in common
amoung all of the two-parameter models described here.

%
These results merely confirm that the theoretical limiting dynamics do indeed
replicate expected RP inference patterns as previously observed in Chapter (\ref{schnuteChapter}).


