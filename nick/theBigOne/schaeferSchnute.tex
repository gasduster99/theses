The Schaefer model is typically stated as discussed in Chapter (\ref{ptChapter})
\begin{equation}
\frac{dB}{dt} = r B \left(1-\frac{B}{K}\right) - FB. \label{sheaferOG}
\end{equation}
However under model \ref{schnuteSimple}, when $\gamma=1$, 
$P_s(B;[\alpha, \beta, 1])$ reduces to quadratic logistic production and 
draws a parallel to the logistic production of the typical Schaefer model. 
%discussed in Chapter (\ref{ptChapter}). The typical formulation, and interpretation, of 
The typical Schaefer model does not explicitly include $M$, but rather natural 
mortality is assumed to be implicit in the quadratic form of surplus production. 
That said, if natural mortality is explicitly included in a model with logistic 
production, as seen in Chapter (\ref{schnuteChapter}) under the Schnute model 
when $\gamma=1$, it is straight forward to show that the result is in fact a 
Schaefer model with updated parameters that explicitly model $M$. 
\begin{align}
\frac{dB}{dt} &= \alpha B \left(1-\beta B\right) - (M+F)B \nonumber\\%. \label{sheaferSchnute}
&= \alpha B -\alpha\beta B^2 - M B - F B \nonumber\\
&= (\alpha-M)B -\alpha\beta B^2 - F B \nonumber\\
&= (\alpha-M)B \left(1 - \frac{\alpha\beta}{\alpha-M}B\right) - F B
\end{align}
%Due to the quadratic form of the Schaefer model 
The linear $MB$ term naturally combines with the quadratic form of 
$P_s(B;[\alpha, \beta, 1])$ to simplify into another quadratic 
form that is now directly analogous with the typical Schaefer model seen in 
Eq. (\ref{sheaferOG}). This process produces a Schaefer model with 
the following updated parameters
\begin{align}
\frac{dB}{dt} &= r' B \left(1-\frac{B}{K'}\right) - FB\\
r'=\alpha&-M ~~~~~~~~~ K'=\left(\frac{\alpha-M}{\alpha}\right)\frac{1}{\beta}.
\end{align}
%
%If instead one were to start in the reverse direction, starting with the 
%typical Schaefer model (with implicte natural mortality) and required
%  
%The maximum rate of surplus biomass production, $r'$,  is the difference between the 
%maximum rate of production, $\alpha$, and $M$ 

%A bit of interpretation. Quadratic production of biomass before Mortality.
$P_s(B;[\alpha, \beta, 1])$ represents the biomass production independent of 
mortality. By combining $MB$ into the quadratic form, a new intrinsic 
population growth rate, $r'$, is derived which represents the maximum rate of 
biomass production less natural mortality. %as a net rate of maximum surplus production. 
Similarly, 
%Maximum rate of production is shrunk by $M$.
carrying capacity is shrunk by the factor $\frac{\alpha-M}{\alpha}$ 
which can be interpreted as the percent decrease of $\alpha$ by removing $M$.
% represents as fraction $\alpha$.
% of the maximum rate of biomass production that decreased by $M$.
%maximum rate of surplus biomass production in maximum production by $M$.

%
Furthermore working with $\gamma=1$ in the analytical expression of Schnute RPs, 
as given in Chapter (\ref{schnuteChapter}), analytical RPs take the expected 
values. For example, working with the equation for $\alpha$ from Eq. (\ref{abgSys}), 
with $\gamma=1$, $F^*$ is shown to takes the expected form for a Schaefer model.
\begin{align}
\alpha &= (M+F^*)\left(1+\frac{F^*}{M+F^*}\right) \nonumber\\
       &= M+2F^* \nonumber\\
   F^* &= \frac{\alpha-M}{2} = \frac{r'}{2} 
\end{align}
%By working with Eq (\ref{BsEq}, \ref{B0S}).
Working with Eq. (\ref{B0S}) from the text gives the expected carrying capacity.
\begin{align}
B_0 &= \frac{1}{\beta} \left( 1-\frac{M}{\alpha} \right) \nonumber\\
    &= \frac{\alpha-M}{\alpha\beta} = K'
\end{align}
Furthermore, Eq. (\ref{B0S}) for $B^*$ gives,

\begin{align}
    B^* &= \frac{1}{\beta} \left( 1-\frac{M+\frac{\alpha-M}{2}}{\alpha}\right) \nonumber\\
        &= \frac{1}{\beta} - \frac{M+\alpha}{2\beta\alpha}\nonumber\\
        &= \frac{\alpha-M}{2\alpha\beta} = \frac{K'}{2}.
\end{align}
%
Finally $\frac{B_{MSY}}{B_0}$ in the case of explicit mortality is indeed also $\frac{1}{2}$,

%\vspace*{-0.5cm}
\begin{align}
\frac{B^*}{B_0}&=\frac{K'/2}{K'}=\frac{1}{2}.
\end{align}
%\vspace*{-0.25cm}

%
This is a similar model as the logistic model discussed by Aalto et. al. \cite{aalto_separating_2015}, 
albeit their model includes additional dynamics. In terms of the complexity of 
dynamics, the model presented by Aalto et. al. prioritizes differing lags of 
recruitment and mortality to increase complexity over the SPM presented in Chapter (\ref{schnuteChapter}). 
%, but the Aalto et. al.  %is intermediate to the SPMs 
%presented in Chapter (\ref{schnuteChapter}) and the DDMs presented in 
%Chapter (\ref{delayChapter}). 
The special case of logistic recruitment in the DDM presented in Chapter (\ref{delayChapter}) 
has a simplified lag structure as compared with the Aalto et. al. model but would be similar 
to their logistic model albeit with an explicit handling of individual growth. 
%
%is more similar to the Aalto et. al. model although prioritizes would produce similar results as the Aalto et. al. model 
%when the recruitment and mortality lagges tied togetherprioritizes an explicit 
%handling of numbers and individual growth to add complexity not included in Aalto et. al.
%
%Similar results may be obtained in the DDM 
%setting here, although without differing recruitment and mortality lags but 
%with an explicit handling of numbers and individual growth.  
